\documentclass[12pt]{report}
\usepackage{graphicx}
\usepackage[utf8]{inputenc}
\usepackage[spanish]{babel}
\usepackage{setspace}
\usepackage{geometry}
\usepackage{titlesec}
\usepackage{times}
\usepackage{mathptmx} % Use mathptmx instead of times
\usepackage{fancyhdr}
\usepackage{float}
\usepackage{pdfpages}



% Configuración de márgenes
\geometry{
    top=2.5cm,
    left=3cm,
    right=3cm,
    bottom=2.5cm
}

% Configuración de interlineado
\onehalfspacing

% Configuración de títulos y subtítulos
\titleformat{\chapter}[display]
  {\normalfont\bfseries\centering}{}{0pt}{\fontsize{14}{16}\selectfont}
\titleformat{\section}
  {\normalfont\bfseries}{\thesection}{1em}{\fontsize{12}{14}\selectfont}
\titleformat{\subsection}
  {\normalfont\bfseries}{\thesubsection}{1em}{\fontsize{12}{14}\selectfont}


% Configuración de pie de página
  \fancyhf{}
\fancyfoot[R]{\thepage}
\pagestyle{fancy}
\fancypagestyle{plain}{
  \fancyhf{}
  \fancyfoot[R]{\thepage}
}

  \begin{document}
  \pagenumbering{roman}
%----- PORTADA ----
\setlength{\hoffset}{27 pt} % 1 (Para centrar más la portada)
\begin{titlepage}
{\centering
{\fontfamily{ptm}\scshape\bfseries\fontsize{29.16}{34.992}\selectfont Universidad de Guadalajara \par}
\vspace{0.5cm}
{\scshape\Large Centro Universitario de los Lagos \par}
\vspace{1cm}
{\scshape\Large División de Estudios de la Biodiversidad e innovación Tecnológica \par}
\vspace{1cm}
{\graphicspath{{imagenes/Portada}} %ruta de las imagenes
\includegraphics[width=0.3\textwidth]{image.png}\par}
\vspace{1cm}
% Título
{\scshape\large\bfseries Practica 6: Función temporizador (TON y TOF) \par}
\vspace{0.5cm}
% Materia
{\large \textbf{Materia:} \\Controladores Lógicos Programables\par}
\vfill
% Estudiante
{\large \textbf{Presenta:} \\Oscar Iván Moreno Gutiérrez \#220942754
\\Maximiliano Frias Campos \#217488066
\par}
\vfill
% Profesor
{\large \textbf{Profesor:} \\Dr. Afanador Delgado Samuel Mardoqueo \par}
\vfill
\vfill
% Fecha
\begin{flushright}
  {\normalsize \textbf {Fecha:} \\ \today}
\end{flushright}
\vfill}
{\large  \par}
\end{titlepage}
%----- FIN DE PORTADA ----

%----- ÍNDICE GENERAL ----
\tableofcontents
\newpage

%----- PALABRAS CLAVE ----
\pagenumbering{arabic}
\chapter*{Palabras Clave}

\addcontentsline{toc}{chapter}{Palabras Clave}
\begin{itemize}
\end{itemize}

%\begin{itemize}

%\end{itemize}
\newpage

%----- OBJETIVO ----
\chapter*{Objetivo}
\addcontentsline{toc}{chapter}{Objetivo}
  El objetivo de esta Practica es utilizar los temporizadores de manera efectiva en la programación de Controladores Lógicos Programables (PLC) para controlar el tiempo de activación y desactivación de una salida. A través de la implementación de temporizadores TON y TOF, se busca simular y verificar el funcionamiento de circuitos de control que requieren un retardo específico para activar o desactivar una salida.

\newpage

%----- CONTENIDO ----
\chapter{Contenido}
\section{¿Qué es un temporizador?}
Un temporizador es un dispositivo que se utiliza para controlar el tiempo de activación y desactivación de una salida en un circuito de control. En la programación de Controladores Lógicos Programables (PLC), los temporizadores se utilizan para introducir retrasos específicos en la activación y desactivación de salidas, lo que permite controlar el tiempo de ejecución de una operación.
\subsection{Funcionamiento de TON}
La función TON (Temporizador ON Delay) se utiliza para activar una salida después de un retardo específico. Cuando se activa la condición de TON, la salida permanece en un estado bajo (apagado) durante un tiempo determinado antes de cambiar a un estado alto (encendido). Una vez que la salida se activa, permanece en ese estado durante un tiempo específico antes de volver a su estado inicial.
\subsection{Funcionamiento de TOF}
La función TOF (Temporizador OFF Delay) se utiliza para desactivar una salida después de un retardo específico. Cuando se activa la condición de TOF, la salida permanece en un estado alto (encendido) durante un tiempo determinado antes de cambiar a un estado bajo (apagado). Una vez que la salida se desactiva, permanece en ese estado durante un tiempo específico antes de volver a su estado inicial.


\section{Materiales}
Para la realización de esta práctica se utilizaron los siguientes materiales:

\begin{itemize}
  \item \textbf{Aplicación con picosoft:} Software utilizado para la simulación y programación de PLCs.
  \item \textbf{PLC:} Controlador Lógico Programable utilizado para la implementación del circuito.
  \item \textbf{Botonera:} Dispositivo que contiene los botones de arranque y paro.
  \item \textbf{Botones:} Componentes individuales de la botonera utilizados para controlar el circuito.
\end{itemize}

\section{Procedimiento}
\begin{enumerate}
  \item Declaramos las variables de nuestro circuito.
        \begin{figure}[H]
          \centering
          \includegraphics[width=0.5\textwidth]{screenshots/variables.png}
          \caption{Variables del circuito}
          \label{fig:variables}
        \end{figure}
  \item Creamos el circuito
        \begin{figure}[H]
          \centering
          \includegraphics[width=1\textwidth]{screenshots/PredominanteSET.png}
          \caption{Circuito dominante SET}
          \label{fig:set}
        \end{figure}
\end{enumerate}
\newpage

%----- CONCLUSIONES ----
\chapter{Conclusiones}
En esta práctica, se implementaron y simularon circuitos de control utilizando las funciones LATCH/UNLATCH, también conocidas como SET/RESET. A través de la creación de variables y la configuración adecuada de los componentes del circuito, se pudo verificar el funcionamiento de los circuitos predominantes SET y RESET.

El circuito predominante SET demostró ser efectivo para mantener una salida activada hasta que se cumpla una condición específica para desactivarla. Por otro lado, el circuito predominante RESET permitió desactivar una salida previamente activada por el circuito SET, asegurando que las operaciones se realicen de manera controlada y segura.
\newpage


\end{document}