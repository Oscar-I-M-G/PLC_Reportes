\documentclass[12pt]{report}
\usepackage{graphicx}
\usepackage[utf8]{inputenc}
\usepackage[spanish]{babel}
\usepackage{setspace}
\usepackage{geometry}
\usepackage{titlesec}
\usepackage{times}
\usepackage{mathptmx} % Use mathptmx instead of times
\usepackage[backend=biber,style=apa]{biblatex} % Add biblatex package
\addbibresource{references.bib} % Name of your .bib file
%%\DeclareLanguageMapping{spanish}{spanish-apa} % Set the language mapping for APA style



% Configuración de márgenes
\geometry{
    top=2.5cm,
    left=3cm,
    right=3cm,
    bottom=2.5cm
}

% Configuración de interlineado
\onehalfspacing

% Configuración de títulos y subtítulos
\titleformat{\chapter}[display]
  {\normalfont\bfseries\centering}{}{0pt}{\LARGE}
\titleformat{\section}
  {\normalfont\bfseries}{\thesection}{1em}{\Large}
\titleformat{\subsection}
  {\normalfont\bfseries}{\thesubsection}{1em}{\large}

  \begin{document}
%----- PORTADA ----
\setlength{\hoffset}{27 pt} % 1 (Para centrar más la portada)
\begin{titlepage}
{\centering
{\fontfamily{ptm}\scshape\bfseries\fontsize{29.16}{34.992}\selectfont Universidad de Guadalajara \par}
\vspace{0.5cm}
{\scshape\Large Centro Universitario de los Lagos \par}
\vspace{1cm}
{\scshape\Large División de Estudios de la Biodiversidad e innovación Tecnológica \par}
\vspace{1cm}
{\graphicspath{{imagenes/Portada}} %ruta de las imagenes
\includegraphics[width=0.3\textwidth]{image.png}\par}
\vspace{1cm}
% Título
{\scshape\large\bfseries Investigación 1. Campos de Aplicación del PLC \par}
\vspace{1.5cm}
% Materia
{\large \textbf{Materia:} \\Controladores Lógicos Programables\par}
\vfill
% Estudiante
{\large \textbf{Presenta:} \\Oscar Iván Moreno Gutiérrez \#220942754\par}
\vfill
% Profesor
{\large \textbf{Profesor:} \\Dr. Afanador Delgado Samuel Mardoqueo \par}
\vfill
\vfill
% Fecha
\begin{flushright}
  {\normalsize \textbf {Fecha:} \\ \today}
\end{flushright}
\vfill}
{\large  \par}
\end{titlepage}
%----- FIN DE PORTADA ----

%----- ÍNDICE GENERAL ----
\tableofcontents
\newpage

%----- PALABRAS CLAVE ----
\chapter*{Palabras Clave}
\addcontentsline{toc}{chapter}{Palabras Clave}
PLC: Controlador Lógico Programable.\\
Industria: Conjunto de procesos y actividades que tienen como finalidad transformar materias primas en productos elaborados.\\
Control: Regulación de un sistema para mantenerlo en un estado deseado.\\
\newpage


%----- OBJETIVO ----
\chapter*{Objetivo}
\addcontentsline{toc}{chapter}{Objetivo}
El objetivo de esta investigación es conocer los campos de aplicación de los PLCs, así como la importancia en la industria. También se pretende conocer las ventajas y desventajas de los PLCs.

%----- CONTENIDO ----
\chapter{Contenido}
\section{Aplicación de PLCs}
La idea general de la utilización de los PLCs es para automatizar procesos de maniobra, control y señalización. Generalmente en la Industria por ejemplo
\begin{itemize}
\item Maquinaria: Máquinas industriales para madera o plásticos, proceso de gravas.
\item Instalaciones: Plantas de embotellado, instalaciones de seguridad, calefacción, tratamientos de agua.
\item Industria Automotriz: Soldaduras, ensamblaje, cabinas de pintura, taladradoras.
\item Industria química y petroquímica: Pesaje, baños eléctricos, dosificación, oleoductos.
\item Metalurgia: Control de hornos, forjas, grúas, laminado.
\item Industria Alimentaria: Empaquetado, almacenaje.
\item Maderas y papeleras: Serradoras, control de procesos, producción de conglomerados.
\item Producción de energía: Energía solar, turbinas eólicas, centrales eléctricas. \cite{Autycom2019}
\end{itemize}
\subsection{Porque usar PLCs}
En las condiciones típicas de una fábrica se trabaja con altas temperaturas y mucho polvo, y los movimientos de las máquinas también causan vibraciones. Los PLCs están diseñados para trabajar en este tipo de condiciones. Además, se busca utilizar el menor costo posible y estos también cumplen, ya que son fáciles de mantener y difíciles de dañar.

\section{Ventajas y Desventajas de los PLCs}
\subsection{Ventajas}
\subsection{Desventajas}



%----- CONCLUSIONES ----
\chapter{Conclusiones}
Aquí van las conclusiones de tu documento.
\newpage

%----- REFERENCIAS ----
\printbibliography
\end{document}